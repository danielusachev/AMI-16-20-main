\documentclass[12pt,a4paper]{scrartcl}
\usepackage[utf8]{inputenc}
\usepackage[english,russian]{babel}
\usepackage{indentfirst}
\usepackage{misccorr}
\usepackage{graphicx}
\usepackage{amsmath}
\newcommand{\task}[1]{
\large {\textbf{#1}}
}
\newcommand*\xor{\mathbin{\oplus}}
\title{Коллок}
\author{Усачев Данила}
\date{\today}
\begin{document}
\maketitle
\noindent
\section*{16}
\section*{Определение}
Булевой схемой от переменных $x_1, \ldots, x_n$ называется последовательность булевых
функций $g_1, \ldots, g_s$, в которой всякая $g_i$ или равна одной из переменных,
или получается из предыдущих применением одной из логических операций отрицание,
конъюнкция и дизъюнкция. Также в булевой схеме задано некоторое число $m \geq 1$
и члены последовательности $g_{s-m+1}, \ldots, g_s$ называются выходами схемы.
Число m называют числом выходов. Число s называют размером схемы.
\section*{Задача}
Пусть на вход подаются переменные $x_{ij}$, $i < j$, означающие, есть ли ребро
между вершинами i и j. Если в графе есть такие вершины i, j, k , $i < j < k$,
которые образуют треугольник, то $x_{ij}x_{jk}x_{ik}=1$. Всего троек вершин
в графе  $C^{n}_{3} = \frac{n(n-1)(n-2)}{6}$.  Тогда схема будет иметь вид
$g = \bar (\bigvee x_{ij}x_{jk}x_{ik}), i \neq j \neq k$ - отрицание дизъюнкции
всех конъюнкций троек вершин. Размер схемы будет равен $O(n^{3})$.\\
\section*{Теорема}
Схема проверки связности графа на n вершинах полиномиального размера.
Пусть матрица $A$ - матрица смежности графа с единицами на главной диагонали.
Можно показать, что на пересечении строки $i$ и столбца $j$ матрицы $A^{k}$
записано число путей длины $k$ из вершины $v_i$ в вершину $v_j$.
Теперь рассмотрим матрицу $A'$, которая отличается от матрицы A тем, что
у нее стоят единицы на главной диагонали.
Заметим следующий факт: если между двумя вершинами есть путь длины меньше n - 1,
то есть и путь длины
ровно $n-1$, достаточно добавить нужное количество петель. То есть надо рассмотреть
матрицу $(A')^{n-1}$. Если в яйчеках нет нулей - граф связен, иначе нет. Теперь
опишем схему.
На вход схема получает матрицу смежности $A'$. Схема последовательно
вычисляет булевы степени этой матрицы $(A')^2,\ldots,(A')^{n-1}$. Затем
схема вычисляет конъюнкцию всех ячеек матрицы $(A')^{n-1}$ и подает ее на выход.\\
Оценим размер схемы. Для булева умножения достаточно $n^2\cdot O(n) = O(n^3)$
операций. Всего нам нужо $(n-1)$ умножений, так что для вычисления матрицы $(A')^{n-1}$
достаточно $O(n^4)$ операций. Для последнего этапа - конъюнкции нужно $O(n^2)$
операций. Итого получается $O(n^4)+O(n^2) = O(n^4)$ операций.
\end{document}
