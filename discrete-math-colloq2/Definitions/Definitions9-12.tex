\documentclass[12pt,a4paper]{scrartcl}
\usepackage[utf8]{inputenc}
\usepackage[english,russian]{babel}
\usepackage{indentfirst}
\usepackage{misccorr}
\usepackage{graphicx}
\usepackage{amsmath}
\newcommand{\task}[1]{
\large {\textbf{#1}}
}
\newcommand*\xor{\mathbin{\oplus}}
\title{Коллок}
\author{Усачев Данила}
\date{\today}
\begin{document}
\maketitle
\noindent

\begin{enumerate}
\item 9 \\
Континуум - мощность множества [0,1]. Примеры:
\begin{enumerate}
\item Множество бесконечных последовательностей нулей и единиц
\item Множество вещественных чисел
\item Квадрат [0,1]x[0,1].
\end{enumerate}


\item 10
\begin {enumerate}
\item В любом континуальном множестве есть счетное подмножество.
\item Мощность объединения не более чем континуального семейства множеств,
каждое из которых не более чем континуально, не превосходит континуума.
\end{enumerate}

\item 11\\
Булева функция от n аргументов - отображение из $B^{n}$ в $B$, где B - $\{0,1\}$.
Количество всех n-арных булевых функций равно $2^{2^{n}}$. Булеву функцию
можно задать таблицей истинности.\\
\item 12\\
Полный базис - это такой набор, который для реализации любой сколь
угодно сложной логической функции не потребует использования каких-либо других
операций, не входящих в этот набор.
Примеры полных базисов:
\begin {enumerate}
\item Конъюнкция, дизъюнкция, отрицание.
\item Конъюнкция, отрицание.
\item Конъюнкция, сложение по модулю два, константа один - базис Жегалкина.
\item Штрих Шеффера (таблица истинности - 0111).
\end {enumerate}

\end{enumerate}


\task{9}\\
Множество бесконечных последовательностей нулей и единиц несчетно. Доказательство:
Предположим, что оно счетно, значит его можно пронумеровать. Тогда построим таблицу
последовательностей.\\
$a_{00} a_{01} a_{02} ...$\\
$a_{10} a_{11} a_{12} ...$\\
$a_{20} a_{21} a_{22} ...$\\
...\\
Теперь рассмотрим диагональную последовательность  $a_{00} a_{11} a_{22} ...$ и
заменим в ней все биты на противоположные. Такая последовательность отличается
от любой $a_{i}$ в i-й позиции, значит этой последовательности нет в списке,
получили противоречие. Значит это множество несчетно.
Теперь докажем, что множество бесконечных последовательностей нулей и единиц равномощно
отрезку [0;1], то есть имеет мощность континуум.
Из курса анализа известно, что каждое число из [0;1] можно представить в виде
бесконечной двоичной дроби. Делается это так: первый бит после запятой равен 0, если
x лежит в левой половине отрезка [0,1] и равен 1, если в правой. И так далее. Делим
отрезок пополам и смотрим, куда попал x. Но это не совсем биекция. Такие последовательности
как 0,1001111... и 0, 101000... соотвествуют одному и тому же числу. Чтобы исправить
это, надо исключить последовательности, в которых начиная с некоторошо момента все
цифры равны 1 (кроме 0.111111...). Но таких последовательностей счетное множество, так
что их добавление не меняет мощность множества.
\end{document}
