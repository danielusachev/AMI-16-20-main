\documentclass[12pt,a4paper]{scrartcl}
\usepackage[utf8]{inputenc}
\usepackage[english,russian]{babel}
\usepackage{indentfirst}
\usepackage{misccorr}
\usepackage{graphicx}
\usepackage{amsmath}
\newcommand{\task}[1]{
\large {\textbf{#1}}
}
\newcommand*\xor{\mathbin{\oplus}}
\title{Коллок}
\author{Усачев Данила}
\date{\today}
\begin{document}
\maketitle
\noindent
\section*{15}
\section*{Определение}
Полином Жегалкина.
Полином Жегалкина — полином с коэффициентами вида
0 и 1, где в качестве произведения берётся конъюнкция, а в качестве сложения
исключающее или. Каждая булева функция единственным образом представляется в виде полинома Жегалкина.
\section*{Задача}
\section*{Теорема}
Cхема умножения n-битовых чисел за $O(n^2)$.
Пусть на вход подаются два числа $x=x_{n-1}\ldots x_1x_0$ и
$y=y_{n-1}\ldots y_1y_0$. Мы хотим вычислить $z = x \cdot y$. Заметим, что
$z$ имеет не больше $2n$ разрядов. Действительно, $x, y < 2^n$, так что
$z = x \cdot y < 2^{2n}$, а значит для его записи достаточно $2n$ разрядов.\\
Для вычисления $z$ воспользуемся школьным методом. В нем умножение двух чисел
сводится к сложению $n$ чисел. Действительно, чтобы умножить $x$ на $y$ достаточно
для всякого $i = 0, \ldots , n - 1$ умножить $x$ на $y_i$, приписать в конце числа
$i$ нулей и затем сложить все полученные числа. Умножение $x$ на $y_i$ легко
реализуется с помощью $n$ конъюнкций. После этого остается сложить $n$ чисел
длины не более $2n$. Для этого мы можем $n-1$ раз применить схему для сложения.
Размер каждой схемы для сложений линейный, так что суммарная сложность схемы для
умножения получается $O(n^2)$.
\end{document}
