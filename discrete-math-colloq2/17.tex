\documentclass[12pt,a4paper]{scrartcl}
\usepackage[utf8]{inputenc}
\usepackage[english,russian]{babel}
\usepackage{indentfirst}
\usepackage{misccorr}
\usepackage{graphicx}
\usepackage{amsmath}
\newcommand{\task}[1]{
\large {\textbf{#1}}
}
\newcommand*\xor{\mathbin{\oplus}}
\title{Коллок}
\author{Усачев Данила}
\date{\today}
\begin{document}
\maketitle
\noindent
\section*{17}
\section*{Определение}
Схемная сложность функции f относительно базиса B —
это минимальное количество функциональных элементов из набора B, необходимое
для реализации функции f в базисе B.
\section*{Задача}
\section*{Теорема}
Разрешимое множество перечислимо.
Алгоритм перечисления множества A использует алгоритм разрешения множества A.
Он перебирает все числа, начиная с 0; для каждого числа n вычисляет индикаторную
функцию $\chi_A(n)$ и печатает число n, если полученное значение равно 1.
Корректность такого алгоритма очевидна из определений.
\end{document}
