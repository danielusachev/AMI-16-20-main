\documentclass[12pt,a4paper]{scrartcl}
\usepackage[utf8]{inputenc}
\usepackage[english,russian]{babel}
\usepackage{indentfirst}
\usepackage{misccorr}
\usepackage{graphicx}
\usepackage{amsmath}

\newcommand{\task}[1]{
\large {\textbf{#1}}
}
\newcommand*\xor{\mathbin{\oplus}}
\title{Коллок}
\author{Усачев Данила}
\date{\today}
\begin{document}
\maketitle
\noindent

\section*{Теорема 9}
Множество [0;1] несчетно и имеет мощность континуум.
Множество бесконечных последовательностей нулей и единиц несчетно. Докажем это.
Предположим, что оно счетно, значит его можно пронумеровать. Тогда построим таблицу
последовательностей.\\
$a_{00} a_{01} a_{02} ...$\\
$a_{10} a_{11} a_{12} ...$\\
$a_{20} a_{21} a_{22} ...$\\
...\\
Теперь рассмотрим диагональную последовательность  $a_{00} a_{11} a_{22} ...$ и
заменим в ней все биты на противоположные. Такая последовательность отличается
от любой $a_{i}$ в i-й позиции, значит этой последовательности нет в списке,
получили противоречие. Значит это множество несчетно.\\
Теперь докажем, что множество бесконечных последовательностей нулей и единиц равномощно
отрезку [0;1], то есть имеет мощность континуум.
Из курса анализа известно, что каждое число из [0;1] можно представить в виде
бесконечной двоичной дроби. Делается это так: первый бит после запятой равен 0, если
x лежит в левой половине отрезка [0,1] и равен 1, если в правой. И так далее. Делим
отрезок пополам и смотрим, куда попал x. Но это не совсем биекция. Такие последовательности
как 0,1001111... и 0, 101000... соотвествуют одному и тому же числу. Чтобы исправить
это, надо исключить последовательности, в которых начиная с некоторошо момента все
цифры равны 1 (кроме 0.111111...). Но таких последовательностей счетное множество, так
что их добавление не меняет мощность множества.
\section*{Теорема 10}
Если для множества A и B существует инъекция из A в B и инъекция из B в A, то
существует и биекция между A и B.
Доказательство. Пусть $f : A \rightarrow B$ и $g : B \rightarrow A$ инъекции.
Рассмотрим орграф с вершинами $A \cup B$. Для точек $x \in A$ и $y \in B$ проводим
ребро из x в y, если $f(x)=y$ и ребро из y в x, если $g(y) = x$. По построению
из каждой точки выходит ровно одно ребро. А так как функции инъективны, то и входит
не больше одного.\\
Разобьем граф на компоненты связности, забыв об ориентации ребер, и рассмотрим каждую
компоненту отдельно. Для каждой компоненты есть три варианта:
Компонента может быть циклом из стрелок, бесконечной цепочкой стрелок, начинающейся
в некоторой вершине или бесконечной в обе стороны цепочкой стрелок.\\
В нашем графе вершины бывают "левые"(из A) и "правые"(из B). Они чередуются,
поэтому цикл может быть только четной длины и содержит поровну вершин из A и из
B. Они чередуются, поэтому цикл может быть только четной длины и содержит поровну
вершин из A и из B. Любое из отображений f и g может быть использовано чтобы
построить биекцию между A и B вершинами цикла. То же самое верно для бесконечной
в обе стороны цепочки. Если же цепочка бесконечна только в одну сторону, то для
построения биекции годится только одно из отображений. Скажем, если она начинается
с a, то нам годится только функция f (при которой a соответствует f(a)).
Но в любом случае, одна из функций f и g годится, так что внутри каждой связной компоненты
у нас есть биекция, и остается их объединить для всех связных компонент.\\
\section*{Теорема 11}
Полнота стандартного базиса.
Любое высказывание может быть выражено как дизъюнкция таких высказываний, у
котороых ровно в одной строке стоит 1, а в остальных стоят нули. Действительно,
выберем все строки таблицы высказывания, в которых стоят единицы. Для каждой
такой строки образуем высказывание, которое истинно только в данной строке, а
в остальных ложно, дизъюнкция всех этих высказываний и будет выражать искомое.\\
Теперь научимся выражать через диъюнкции, конъюнкции и отрицания высказывания
того вида, который использован в предыдущей конструкции. Чтобы получить высказывание
, которое истинно ровно для одного произвольного набора логических значений, сделаем
следующее: Если значение какой-то переменной равно единице, то включим эту переменную
в высказывание, а если нулю, то включим ее отрицание. Построенная конъюнкция
принимает значение 1 лишь тогда, когда все ее члены равны 1. По построению это
происходит ровно на одном наборе значений переменных.
\section*{Теорема 12}
Существование и единственность полинома Жегалкина.
Сначала докажем по индукции, что любое произвольное высказывание $f(x_1 \ldots x_n)$
можно выразить формулой со связкам $\wedge, \xor, 1$. База индукции $n = 1$.
Константа 1 уже есть. 0 выражается как $1 \xor 1$. \\
Пусть утверждение доказано для всех составных высказываний от $n$ элементарных
высказываний. Докажем выразимость для составных высказываний от $n + 1$ элементарного
высказывания. Для этого по высказыванию $f(x_1, \ldots , x_n+1)$ определим
два высказывания от $n$ элементарных высказываний, а именно $f_0(x_1, \ldots x_n) =
f(x_1, \ldots , x_n, 0) $ и $f_1(x_1, \ldots , x_n) = f(x_1, \ldots , x_n, 1)$.
По предположению индукции $f_0$ и $f_1$ выражаются через базис Жегалкина. Выразим
теперь f (разложение Рида): $f = ((1 \xor x_{n+1}) \wedge f_0) \xor (x_{n+1} \wedge f_1)$.
Действительно, при $x_{n+1} = 0$ обращается в 0 второе слагаемое, при $x_{n+1} = 1$ -
первое. В любом случае получаем совпадение левой и правой частей равенства.
\\
Теперь докажем единственность.
Заметим, что различных булевых функций от n переменных $2^{2^n}$ штук.
При этом конъюнкций вида $x_{i_1} \ldots x_{i_k}$ существует ровно $2^n$, так как из
n возможных сомножителей каждый или входит в конъюнкцию, или нет. В полиноме у
каждой такой конъюнкции стоит 0 или 1, то есть существует $2^{2^n}$ различных
полиномов Жегалкина от n переменных.\\
Теперь достаточно лишь доказать, что различные полиномы реализуют различные функции.
Предположим противное. Тогда приравняв два различных полинома и перенеся один из них
в другую часть равенства, получим полином, тождественно равный нулю и имеющий ненулевые
коэффициенты. Тогда рассмотрим слагаемое с единичным коэффициентом наименьшей длины,
то есть с наименьшим числом переменных, входящих в него (любой один, если таких несколько).
Подставив единицы на места этих переменных, и нули на места остальных, получим,
что на этом наборе только одно это слагаемое принимает единичное значение, то
есть нулевая функция на одном из наборов принимает значение 1. Противоречие.
Значит, каждая булева функция реализуется полиномом Жегалкина единственным образом.
\end{document}
